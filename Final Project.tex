\documentclass[11pt]{article}

\usepackage{amsmath,setspace,mathtools,amssymb,booktabs,graphicx, multicol, bm}

\usepackage[utf8]{inputenc}

\usepackage[letterpaper,portrait,margin=0.5cm]{geometry}

\graphicspath{ {./} }


\title{BPhys450 Final Project}

\author{Thaseus Karkabe-Olson}

\date{}

\begin{document}

	\maketitle

    \raggedright

    \section*{Goals}

        The goal of this computational model is to enhance student understanding in a preclass activity. As such, it should be tailored to the learning goals of previous preclass activities. Using the current list of learning goals for magnatism in BPHYS122 as a reference, students should be able to:

        \begin{itemize}

            \item Describe similarities and differences between electric and magnetic fields
            \item Use a compass to identify the direction and relative magnitude of a magnetic field, and to identify the north pole of a bar magnet
            \item Draw and interpret magnetic field vectors and field lines: (1) in general, (2) for a bar magnet, (3) for a straight current-carrying wire, and (4) for a loop of current
            \item Calculate the magnetic force on (1) a charged particle moving in a magnetic field, and (2) on a current-carrying wire in a magnetic field
            \item Explain how a charged particle in an external magnetic field undergoes circular motion, and find the radius of that motion
            \item Calculate the force on a current-carrying wire in an external magnetic field (magnitude and direction)
            \item Evaluate the net force on a current loop in an external magnetic field
            \item Evaluate the net torque on a current loop in an external magnetic field
            \item Define the magnetic dipole moment of a current loop

        \end{itemize}

        For this final project I will focus on creating a preliminary active learning activity with an integrated model that addresses goals 3, 4, 5, and (6?)


\end{document}
