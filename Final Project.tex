\documentclass[11pt]{article}

\usepackage{amsmath,setspace,mathtools,amssymb,booktabs,graphicx, multicol, bm}

\usepackage[utf8]{inputenc}

\usepackage[letterpaper,portrait,margin=0.5cm]{geometry}

\graphicspath{ {./} }


\title{BPhys450 Final Project}

\author{Thaseus Karkabe-Olson}

\date{}

\begin{document}

	\maketitle

    \raggedright
    
    \section*{Abstract}

        The goal of this computational model is to enhance student understanding in a preclass activity for a given week. As such, it should be tailored to the learning goals of previous preclass activities. Using the current list of learning goals for week 7 magnatism in BPHYS122 as a reference, students should be able to:

        \begin{itemize}

            \item Describe similarities and differences between electric and magnetic fields
            \item Use a compass to identify the direction and relative magnitude of a magnetic field, and to identify the north pole of a bar magnet
            \item Draw and interpret magnetic field vectors and field lines: (1) in general, (2) for a bar magnet, (3) for a straight current-carrying wire, and (4) for a loop of current
            \item Calculate the magnetic force on (1) a charged particle moving in a magnetic field, and (2) on a current-carrying wire in a magnetic field
            \item Explain how a charged particle in an external magnetic field undergoes circular motion, and find the radius of that motion
            \item Calculate the force on a current-carrying wire in an external magnetic field (magnitude and direction)
            \item Evaluate the net force on a current loop in an external magnetic field
            \item Evaluate the net torque on a current loop in an external magnetic field
            \item Define the magnetic dipole moment of a current loop

        \end{itemize}

        For this final project I will focus on creating a preliminary active learning activity with an integrated model that addresses goals 4a and 5. The experiment I conduct will be figuring out how a non-physics person interacts with the model, in an effort to identify if I can use the model to teach basic magnetism.
        
   	    \section*{Program Design}
   	    
   	    \section*{Activity}
   	    
   	    As this is focused (at the moment) on qualitative understanding, the questions I present will for the most part be pretty open ended. Here is the activity that I designed:
   	    
   	    \subsection*{Introduction}
   	    
   	    	In this activity you will interact with a model. This model involves a particle with mass, velocity, and charge. When the experiment is run, the particle will travel through empty space for some time. Then a magnetic field pointing either into the page (marked with Xs) or out of the page (marked with dots) will appear for some time. Then the field will turn off.
   	    
   	    \subsection*{Question 1}
			Change some parameters and run some experiments. Write down 5 observations that you noticed.
			
		\subsection*{Question 2}
			\subsubsection*{Part a}
				What are some ways that you can get the particle to travel in a straight line?
				
			\subsubsection*{Part b}
				For each method you listed, why do you think the particle traveled in a straight line?
			
		\subsection*{Question 3}
			Sometimes the particle travels along a curve. How can you relate each of the properties given by the model to the radius of that curve?
			
		\subsection*{Question 4}
			What are two questions you have about the system?

\end{document}
