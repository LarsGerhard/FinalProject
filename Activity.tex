\documentclass[11pt]{article}

\usepackage{amsmath,setspace,mathtools,amssymb,booktabs,graphicx, multicol, bm}

\usepackage[utf8]{inputenc}

\usepackage[letterpaper,portrait,margin=0.5cm]{geometry}

\graphicspath{ {./} }


\title{Magnetic Field Activity}

\author{}

\date{}

\begin{document}
	
	\maketitle
	
	\raggedright
	
	
	\section*{Introduction}
	
	In this activity you will interact with a model. This model involves a particle with mass, velocity, and charge. When the experiment is run, the particle will travel through empty space for some time. Then a magnetic field pointing either into the page (marked with Xs) or out of the page (marked with dots) will appear for some time. Then the field will turn off.
	
	\section*{Question 1}
	Change some parameters and run some experiments. Write down 5 observations that you noticed. \\[180 pt]
	
	\section*{Question 2}
	\subsection*{Part a}
	What are some ways that you can get the particle to travel in a straight line? \\[100 pt]
	
	\subsection*{Part b}
	For each method you listed, why do you think the particle traveled in a straight line? \\[100 pt]
	
	\section*{Question 3}
	Sometimes the particle travels along a curve. How can you relate each of the properties given by the model to the radius of that curve? \\[150 pt]
	
	\section*{Question 4}
	What are two questions you have about the system? \\[150 pt]
	
\end{document}